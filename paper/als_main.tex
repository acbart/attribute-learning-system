% For tracking purposes - this is V3.1SP - APRIL 2009

\documentclass{acm_proc_article-sp}

\usepackage{alltt}
\renewcommand{\ttdefault}{txtt}

\begin{document}

\title{Attribute Learning System}
\subtitle{[Applying Genetic Algorithms to Improve RPG Combat Mechanics]} % Dude that title sucks


\numberofauthors{2} 

\author{
% 1st. author
\alignauthor
Austin Cory Bart\\
       \affaddr{Virginia Tech}\\
       \email{acbart@vt.edu}
% 2nd. author
\alignauthor
K. Alnajar\\
       \affaddr{Virginia Tech}\\
       \email{kar@vt.edu}
\alignauthor
}

\date{24 April 2013}

\maketitle
\begin{abstract}
Having a good set of moves for players to choose from in role-playing games (RPG) is essential for the game to succeed.  Often times in an RPG, the players have various attributes which these moves can effect and coming up with good formulas for this is not easy. The process of creating an effective set of moves can take time and can be a difficult challenge to overcome in the design process. This paper propeses an implementation to effectively create these moves using a genetic algorithm implementation. Two seperate implementation styles of genetic algorithms are used, a tree style and a vector style. The results show that the vector styled approach for the genetic algorithm shows promising results in move set creation.
\end{abstract}

% A category with the (minimum) three required fields
% TODO: We need to look up the tags for this research area
\category{1.2.1}{ARTIFICIAL INTELLIGENCE}{Applications and Expert Systems --- \textit{games}}

\terms{Genetic Programming}

\keywords{Genetic, Programming, Game, Development} % NOT required for Proceedings

\section{Problem}
	%The Computer Science Education community has an established interest in skills \cite{bayliss:csgames}.
	
\section{Approach}

    \subsection{Prior Work}

    \subsection{Target Audience}

\section{Implementation}

    \subsection{Genetic Algorithm}

    \subsection{Function Tree}

        \subsubsection{Mutation Algorithm}
        \subsubsection{Cross-over Algorithm}

%    Mixed-children variant - http://ieeexplore.ieee.org.ezproxy.lib.vt.edu:8080/stamp/stamp.jsp?tp=&arnumber=6256587
    Weighted-delay cross-over variant - Potentially novel
    Mixed-children weighted-delay cross-over

    \subsection{Function Vector}
    
        \subsubsection{Mutation Algorithm}
        \subsubsection{Cross-over Algorithm}

    \subsection{Simulation}
    
    describe the algorithm

    \subsection{Fitness Function}
    
    Tried several approaches:
    \begin{itemize}
        \item Move usage**
        \item Battle victory**
        \item Battle length
        \item Linearity
    \end{itemize}
    
    

    \subsection{Players}

        \subsubsection{Minimax Player}
    
        \subsubsection{Greedy Player}
    
        \subsubsection{Random Player}

        \subsubsection{Utility calculation}
        
            Primaries vs. Primaries + Secondaries
            
\section{Validation}

    \subsection{Function Tree}
    
        \subsubsection{Levenshtein Edit Distance}
        
            \paragraph{Mutation}
            
            Original attempt was to simply randomly change nodes in the tree.
            
            \paragraph{Cross-over}
            
        \subsubsection{Numerical Analysis}
        
            \paragraph{Mutation}
            \paragraph{Cross-over}
            
\section{Results}
    
    \subsection{Parameters}

\section{Conclusion}

	We need to do work!
    
\section{Future Work}

    \subsection{Genetic Operators}


%
% The following two commands are all you need in the
% initial runs of your .tex file to
% produce the bibliography for the citations in your paper.
\bibliographystyle{abbrv}
\bibliography{sigproc}  % sigproc.bib is the name of the Bibliography in this case
% You must have a proper ".bib" file
%  and remember to run:
% latex bibtex latex latex
% to resolve all references

\begin{thebibliography}{1}

% Bibliography goes here

\end{thebibliography}

%Generated by bibtex from your ~.bib file.  Run latex,
%then bibtex, then latex twice (to resolve references)
%to create the ~.bbl file.  Insert that ~.bbl file into
%the .tex source file and comment out
%the command \texttt{{\char'134}thebibliography}.

\balancecolumns
% That's all folks!
\end{document}
