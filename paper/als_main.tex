% For tracking purposes - this is V3.1SP - APRIL 2009

\documentclass{acm_proc_article-sp}

\usepackage{alltt}
\renewcommand{\ttdefault}{txtt}

\begin{document}

\title{Attribute Learning System}
\subtitle{[Using Genetic Programming to Find Fun Movelists]} % Dude that title sucks
%
% You need the command \numberofauthors to handle the 'placement
% and alignment' of the authors beneath the title.
%
% For aesthetic reasons, we recommend 'three authors at a time'
% i.e. three 'name/affiliation blocks' be placed beneath the title.
%
% NOTE: You are NOT restricted in how many 'rows' of
% "name/affiliations" may appear. We just ask that you restrict
% the number of 'columns' to three.
%
% Because of the available 'opening page real-estate'
% we ask you to refrain from putting more than six authors
% (two rows with three columns) beneath the article title.
% More than six makes the first-page appear very cluttered indeed.
%
% Use the \alignauthor commands to handle the names
% and affiliations for an 'aesthetic maximum' of six authors.
% Add names, affiliations, addresses for
% the seventh etc. author(s) as the argument for the
% \additionalauthors command.
% These 'additional authors' will be output/set for you
% without further effort on your part as the last section in
% the body of your article BEFORE References or any Appendices.

\numberofauthors{2} %  in this sample file, there are a *total*
% of EIGHT authors. SIX appear on the 'first-page' (for formatting
% reasons) and the remaining two appear in the \additionalauthors section.
%
\author{
% You can go ahead and credit any number of authors here,
% e.g. one 'row of three' or two rows (consisting of one row of three
% and a second row of one, two or three).
%
% The command \alignauthor (no curly braces needed) should
% precede each author name, affiliation/snail-mail address and
% e-mail address. Additionally, tag each line of
% affiliation/address with \affaddr, and tag the
% e-mail address with \email.
%
% 1st. author
\alignauthor
Austin Cory Bart\\
       \affaddr{Virginia Tech}\\
       \email{acbart@vt.edu}
% 2nd. author
\alignauthor
K Aljanar\\
       \affaddr{Virginia Tech}\\
       \email{karvt.edu}
\alignauthor
}
% There's nothing stopping you putting the seventh, eighth, etc.
% author on the opening page (as the 'third row') but we ask,
% for aesthetic reasons that you place these 'additional authors'
% in the \additional authors block, viz.
\date{24 April 2013}
% Just remember to make sure that the TOTAL number of authors
% is the number that will appear on the first page PLUS the
% number that will appear in the \additionalauthors section.

\maketitle
\begin{abstract}
Blah blah blah abstract goes here.
\end{abstract}

% A category with the (minimum) three required fields
% TODO: We need to look up the tags for this research area
\category{K.3.1}{COMPUTERS AND EDUCATION}{Computer Uses in Education}
%A category including the fourth, optional field follows...
\category{K.3.2}{COMPUTERS AND EDUCATION}{Computer and Information Science Education}

\terms{Genetic Programming}

\keywords{Genetic, Programming, Game, Development} % NOT required for Proceedings

\section{Problem Statement}
	%The Computer Science Education community has an established interest in skills \cite{bayliss:csgames}.
    
	
	\subsection{More details}

    Blahdyblah

\begin{itemize}
	\item This is a list
	\item And there's even some \textbf{bold text} in this list.
	\item that's enough listing.
\end{itemize}

    Pancakes.
	
\section{Solution}

\subsection{Prior Work}

Need to edit this
	
\subsection{Target Audience}
	
But I'm so tired 

\begin{description}
	\item[Sleepiness] The state that I'm currently in
	\item[Laziness] I'm like this way too often
	\item[Hungry] I should probably eat soon.
\end{description}

% system is described in Figure \ref{fig:architecture}.
	
%\begin{figure}
%\centering
%\epsfig{file=filename_without_extension_so_can_have_both_eps_and_png, width=3in}
%\label{fig:architecture} % <--- internal label
%\caption{This text appears under the figure}
%\end{figure}
	

%\begin{listings}
%Listings are pretty cool. It's how you embed code. But I forgot to include the package.
%\end{listings}

\section{Future Work}

    \subsection{More work}
    
    \subsubsection{WE NEED TO GO DEEPER}

    This needs work

\section{Conclusion}

	Blather

%
% The following two commands are all you need in the
% initial runs of your .tex file to
% produce the bibliography for the citations in your paper.
\bibliographystyle{abbrv}
\bibliography{sigproc}  % sigproc.bib is the name of the Bibliography in this case
% You must have a proper ".bib" file
%  and remember to run:
% latex bibtex latex latex
% to resolve all references

\begin{thebibliography}{1}

% Bibliography goes here

\end{thebibliography}

%Generated by bibtex from your ~.bib file.  Run latex,
%then bibtex, then latex twice (to resolve references)
%to create the ~.bbl file.  Insert that ~.bbl file into
%the .tex source file and comment out
%the command \texttt{{\char'134}thebibliography}.

\balancecolumns
% That's all folks!
\end{document}
